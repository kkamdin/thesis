% Abstract

%\renewcommand{\abstractname}{Abstract} % Uncomment to change the name of the abstract

\pdfbookmark[1]{Abstract}{Abstract} % Bookmark name visible in a PDF viewer

\begingroup
\let\clearpage\relax
\let\cleardoublepage\relax
\let\cleardoublepage\relax

\chapter*{Abstract}

\begin{center}
\myTitle \\ \bigskip
by \\ \bigskip
\myName \\ \bigskip
Doctor of Philosophy \\ \smallskip
University of California, Berkeley \\ \smallskip
Professor Daniel McKinsey, Chair \\
\end{center}

\vspace{2cm}

\noindent The nature of dark matter is one of the most compelling mysteries of modern physics. Liquid xenon detectors have been at the forefront of the attempt to directly detect dark matter particles for the last decade. The Large Underground Xenon (LUX) experiment recently concluded its operations at the Sanford Underground Research Facility in Lead, South Dakota. During its tenure, LUX set world-leading limits on WIMP dark matter, and paved the way for novel calibration techniques. Due in part to experiments like LUX, the available WIMP parameter space has been dwindling, making experimentalists look to other dark matter candidates. While new technologies will no doubt be needed in the hunt for dark matter, the well-understood detector technology of liquid xenon can be leveraged to search for non-WIMP dark matter. One such candidate, from the general class of dark sector theories, is the Lightly Ionizing Particle (LIP). LIPs interact with regular matter with an effective fractional charge, and the LUX detector is capable of detecting such interactions. New analysis techniques were developed to search for cosmogenic LIPs in the LUX detector, the first such analysis of its kind in liquid xenon. 

Larger, more sensitive successors to the LUX experiment are already in planning and construction phases. As detectors become more sensitive, previously subdominant effects become more important. Under the umbrella of detector R\&D for LUX's more sensitive successor, LZ, a test bed was built to investigate issues that may threaten the sensitivity of future xenon dark matter detectors. Radon backgrounds and the phenomenon of delayed electron noise were studied, revealing new behavior that may help control backgrounds. Delayed electron noise is of special interest, as this phenomenon is the main background for the LIP search.

\endgroup			

\vfill
