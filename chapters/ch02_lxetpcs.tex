%************************************************
\chapter{Particle Detection with Liquid Xenon}

\label{ch:LXeTPCs} 
%************************************************

\section{Liquid Xenon as a Detector Medium}
Liquid xenon detectors are powerful tools for rare event searches. In particular, the dual phase \ac{LXe} \ac{TPC} has been very successful in accessing \ac{WIMP} parameter space and currently holds the worlds most sensitive limits on \ac{WIMP}s. This section describes the properties of \ac{LXe} and the basic principles of \ac{TPC}s that have allowed this technology to play a large role in the hunt for dark matter.

\subsection{Properties of Liquid Xenon}
Liquid xenon has many properties relevant to particle detection, particle identification, and also many properties related to the ease of detector operation:

\begin{itemize}
  \item The density of \ac{LXe} is 2.9~g/cm$^{3}$ at 170~K, much denser than other possible \ac{TPC} target materials, such as liquid argon which has density 1.4~g/cm$^{3}$ at 87~K. The advantage in this two-fold: (1) the same volume contains more kg of Xe than Ar, so for two detectors of the same volume, one filled with Xe and the other with Ar, both running for one year, the Xe detector has more exposure; (2) xenon's high density effectively stops external radiation, producing an ultra-low-background volume in the center of the detector where rare-event searches can be performed (this region is called the ``fiducial volume'').
  
  \item Xenon gas is easily liquefied with liquid nitrogen (77~K) or commercially available pulse tube refrigerators.

  \item Xenon has no long-lived radioisotopes that cause troublesome backgrounds. The one exception is the 2$\nu\beta\beta$ decay of $^{136}$Xe (natural abundance 8.875\%) with measured half-life of $2.1 \times 10^{21}$ years. The long half-life and relatively low abundance together result in a very low count rate, and the isotope can be used to search for neutrino-less double beta decay ($0\nu\beta\beta$).
  
  \item Xenon, as a noble element, is easily purified with a heated getter to rid electronegative impurities (e.g. O$_{2}$) that interfere with the ionization signal. 
  
  \item The comparatively large mass of xenon allows it to be purified of other noble gasses via gas chromatography \cite{LUXKrRemoval2018} and cryogenic distillation \cite{Xe1TKrRemoval2017}. As other noble gasses cannot be removed via getter, this feature is extremely useful in removing the troublesome background of $^{85}$Kr decay. $^{85}$Kr decays via beta emission to stable $^{85}$Rb with a half-life of 10.8 years and Q$_{\beta}$ = 687~keV. The decay proceeds directly to the $^{85}$Rb ground state with a branching ratio of 99.6\%. Since no de-excitation of $^{85}$Rb follows, this beta decay cannot be rejected as background by coincidence with a gamma, and relies purely on the ability to discriminate between WIMP-like \ac{NR} and beta- or gamma- produced \ac{ER}. While ER/NR discrimination is one of the features of \ac{LXe} \ac{TPC}s (described in section \ref{sec:er_nr_discrimination}), leakage of \ac{ER} events in to the \ac{NR} signal region can occur and the best mitigation is to remove as much of the  $^{85}$Kr as possible. Single-phase \ac{LXe} detectors, with no ER/NR discrimination, benefit greatly from the ability to remove $^{85}$Kr.
  
  \item Particles interacting in \ac{LXe} excite atoms and create electron ion-pairs, producing detectable quanta: scintillation photons and ionization electrons, respectively (described in section \ref{sec:signal_generation}).
  
  \item Xenon produces scintillation light of wavelength $\lambda$~=~178~nm (described in section \ref{sec:signal_generation}). Xenon is transparent to this wavelength so it can propagate freely and be directly detected with current \ac{PMT} technology, and doesn't require the use of e.g wavelength shifter. 
  
  \item Ionization electrons produced in particle interactions can be drifted and extracted into a gaseous region via applied electric fields, where they undergo proportional scintillation. By this method, a single electron is amplified many-fold into detectable photons. This basic operating principle of dual-phase \ac{TPC}s makes even a single ionization electron detectable. 
  
  \item Xenon has high light and charge yields, and therefore a low threshold for producing detectable quanta. A useful quantity is the so-called `W-value' of \ac{LXe}: W = 13.7 $\pm$ 0.2 eV \cite{Dahl2009}. The W-value, analogous to a work-function, is a measure of the average energy expenditure to produce one quanta (scintillation photon or an ionization electron) from liquid xenon. 
  
  \item \ac{LXe} \ac{TPC}s are easily scalable: creating a large homogenous volume is straightforward. In contrast, solid state detectors, such as cryogenic Ge, are more difficult to scale up directly and require instead the production of multiple small modules (O(10)~cm) which each must be instrumented separately.  
    
\end{itemize}


\subsection{Scintillation and Ionization Signal Generation}
\label{sec:signal_generation}
A particle can interact with a xenon atom through interaction with an orbiting electron, creating an \ac{ER}, or though an interaction with the xenon nucleus, where the nucleus is imparted with momentum and recoils, \ac{NR}. Some energy is lost to atomic motion (heat). The recoiling electron or nucleus loses energy via interaction with neighboring xenon atoms, creating more excited atoms and electron-ion pairs. The excited xenon atoms, Xe$^{*}$, combine with other atoms to form an excited dimer, or excitons, Xe$_{2}^{*}$. The excited dimer forms two states: a triplet and a singlet, which de-excite with the emission of a 178~nm photon. The lifetimes of the triplet and singlet are measured to be 24~ns and 3~ns, respectively \cite{Mock2014}. The Xe$^{+}$ ions of the electron-ion pairs combine with other Xe atoms to form dimers Xe$_{2}^{+}$, and these dimers can combine with electrons (from the electron-ion pairs) to form excitons, Xe$_{2}^{*}$, which then decay and produce additional 178~nm scintillation photons. This process is called recombination. If no electric field is applied, all electron-ion pairs recombine to produce additional scintillation photons. If an external electric field is present, some electrons can be drifted away from the interaction site to be detected with other methods. 

The sensitivity of liquid xenon detectors to low energy recoils depends on their ability to detect the 178~nm scintillation photons with high-efficiency. High \ac{QE} \ac{PMT}s constructed with ultra-low radioactivity materials are the go-to instrument for this purpose. In addition to high-efficiency photon-detectors, liquid xenon detectors must also have high geometrical light collection efficiency to optimize sensitivity. Single-phase liquid xenon detectors, where no electric field is applied, maximize light-collection by with a spherical geometry, endeavoring to cover 4$\pi$ steradians surrounding the \ac{LXe}. The XMASS detector uses spherical geometry to accomplish photocathode coverage of \~62\%, and two types of Hamamatsu \ac{PMT}s (R10789-11 and R10789-11MOD) with \ac{QE} of 28\%, and quote a signal collection efficiency of 20\% \cite{Abe2013}, \cite{XMASSCollaboration2018}. Dual-phase \ac{TPC} detectors are lined with \ac{PTFE} to take advantage of its extremely high (~99\%) reflectivity for 178~nm light in \ac{LXe} \cite{Neves2017}. The LUX detector uses a cylindrical geometry, with all non-light-collectiing surfaces lined with \ac{PTFE}, and Hamamatsu R8778 \ac{PMT}s (\ac{QE} of 33\%) only on the top and bottom of the detector (low photocathode coverage), to accomplish a light collection efficiency of 90\% \cite{Faham2014}.

If the detector is a \ac{TPC} the ionization electrons are drifted away from the interaction site to be detected. Single phase \ac{TPC} employ thin wires to collect the ionization electrons. For example, the EXO-200 experiment is a single-phase liquid \ac{TPC} that uses crossed-wire planes to collection ionization electrons and avalanche photodiodes to collect the scintillation photons \cite{Auger2012}. LUX is a dual-phase \ac{TPC}, where ionization electrons are drifted and extracted into gaseous xenon via applied electric fields, where they undergo proportional scintillation. The proportional scintillation light is the same 178~nm wavelength as scintillation in the liquid region, and it is similarly collected via high \ac{QE} \ac{PMT}s.


\section{Dual-Phase Liquid Xenon Time Projection Chamber}
\section{ER, NR Discrimination}
\label{sec:er_nr_discrimination}

\section{LUX}
\subsection{Optimization for WIMPs}
\subsection{WIMP Rates and Cross section}
\subsection{LIP Search Ability}


%*****************************************
%*****************************************
%*****************************************
%*****************************************
%*****************************************

