%************************************************
\chapter{Summary and Outlook}\label{ch:conclusion} % $\mathbb{ZNR}$
%************************************************

%\begin{flushright}{\slshape    
 %   I was dreamin' when I wrote this, forgive me if it goes astray. } \\ \medskip
 %   --- {Prince, \textit{1999}, 1989}
%\end{flushright}

The nature of dark matter remains one of the most compelling mysteries of modern physics, and the \ac{LUX} experiment has been noteworthy in setting limits on available \ac{WIMP} dark matter parameter space. Although \ac{WIMP}s are a dark matter candidate well-motivated by cosmology and particle physics, other classes of dark matter candidates, namely from the dark sector, are also excellent candidates. The search for \ac{LIP}s in this thesis explored the possible parameter space available for one class of dark photon model. \ac{LIP}s arise when the dark photon is massless and other dark sector particles gain an effective fractional charge in their interaction with the \ac{SM}. The \ac{LIP} search presented in Chapter~\ref{ch:lips} was successful in putting further limits on the available parameter space for cosmogenic \ac{LIP}s. The analysis pioneered a new technique for \ac{LXe} \ac{TPC}s, in which track reconstruction and energy consistency variables both play a role. 

One limitation of the \ac{LIP} search was in the data processing: for low charge fractions ($f \lesssim 20$), \ac{LIP}s were found to deposit so much energy that their signals produced ionization trails, which the default \ac{LUX} \ac{DPF} was unable to process into individual pulses. One way to drastically improve the \ac{LIP} sensitivity at low $f$ is to implement new data processing methods capable of dealing with continuous ionization signals. The main background for the \ac{LIP} search is the well-known, troublesome phenomenon of electron trains. These delayed electron backgrounds continue for $O(10-100)$~ms, taking up valuable detector livetime. Electron trains are typically considered an annoyance for \ac{WIMP} searches, but they severely limit the sensitivity of \ac{LXe} \ac{TPC}s to some dark matter candidates such as low-mass \ac{WIMP}s or massive dark photons (see Chapter~\ref{sec:non_wimp_searches_with_lxetpcs}), and even, as we saw in Chapter~\ref{ch:lips}, \ac{LIP}s with charge fractions sufficiently high ($f \gtrsim 400$) to dominantly produce SE pulses instead of S2 pulses. 

Chapter~\ref{ch:etrains} of this thesis describes the current understanding of the causes of electron trains, and details a study that revealed two distinct time components in electron trains: a slow component and a fast component. This study was carried out on a laboratory test bed built over the course of this thesis, and showed that the slow component of electron trains is tied to \ac{LXe} purity. It is the slow component of electron trains that extends far past a single event window, polluting future events with spurious signals. That the slow component can be reduced with improved purity holds promise for future \ac{LXe} \ac{TPC}s, as improvements to \ac{LXe} cleaning methods can reduce or possibly eliminate the long time component of electron trains.

Another particularly troublesome background for \ac{LXe} \ac{TPC}s comes from $^{222}$Rn and its daughters. In fact, $^{222}$Rn emanation is projected to produce the majority of the \ac{ER} background counts in \ac{LZ} \cite{LZ:Sensitivity}; some of these \ac{ER} events can leak into the \ac{NR} band, producing spurious \ac{WIMP} signals. A new possible source of background counts in the fiducial volume is highlighted in Chapter~\ref{ch:radon} of this thesis. The practice in the field has historically been to assume the $^{222}$Rn late chain naked beta decays of $^{210}$Pb and $^{210}$Bi are fixed to a surface of the detector and can be rejected via a fiducial cut. The studies carried out in this thesis provide evidence that either $^{212}$Pb or $^{212}$Bi (from the $^{220}$Rn late chain) are mobile to a small degree in \ac{LXe} after having been fixed on a surface. The measurement in this thesis (from $^{220}$Rn) shows considerable disagreement with limits from \ac{LUX} (from $^{222}$Rn), which can be accounted for if beta recoil is the mechanism for radon daughter mobility. Further study with samples of known surface area and activity should be carried out with both $^{222}$Rn and $^{220}$Rn to identify 

The next decade is an exciting time for direct detection. \ac{LZ}, the ton-scale successor to \ac{LUX}, is currently under construction and is expected to reach a best sensitivity of $1.6 \times 10^{-48}$~cm$^{2}$ (for a 40~GeV \ac{WIMP}) for 1000 live days \cite{LZ:Sensitivity}. \ac{LZ} will push further down into \ac{WIMP} parameter space, inching closer to the ``neutrino floor,'' where neutrinos from solar, atmospheric, and astrophysical sources start appearing as background in direct detection experiments. In fact, \ac{LZ} is expected to see nuclear recoil events from $^{8}$B solar neutrinos; these events appear in the low-mass \ac{WIMP} signal region and are taken into account in the \ac{WIMP} search analysis \cite{LZ:Sensitivity}. With new analysis techniques and technology updates, \ac{LXe} detectors are poised to make an impact even in non-\ac{WIMP} dark matter searches. R\&D to create a \ac{LXe} \ac{TPC} that is free of delayed electron backgrounds is underway with the \ac{LBECA} experiment. Even a 10~kg \ac{LXe} detector has discovery potential for massive dark photons if electron backgrounds can be mitigated \cite{cosmicvisions2017}. 

New detector technology will push the direct detection boundaries to new theories and new signals. Whether or not upcoming experiments will uncover any hints of elusive dark matter remains to be seen.


%*****************************************
%*****************************************
%*****************************************
%*****************************************
%*****************************************
