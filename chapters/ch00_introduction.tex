%************************************************
\chapter{Introduction}\label{ch:introduction} % $\mathbb{ZNR}$
%************************************************

\begin{flushright}{\slshape    
    I was dreamin' when I wrote this, forgive me if it goes astray. } \\ \medskip
    --- {Prince, \textit{1999}, 1989}
\end{flushright}

This thesis is laid out in four main sections. 

Part~\ref{part:intro}, Theoretical Context and Experimental Strategies, deals with the evidence for dark matter and theoretical background for different dark matter candidates. An overview of experimental strategies to detect dark matter is then given, going into detail for one particular detection strategy, the dual-phase \ac{LXe} \ac{TPC}.

Part~\ref{part:lux}, Big Science, delves into details about the \ac{LUX} detector, a dual-phase \ac{LXe} \ac{TPC}, and presents a search for the \ac{LIP} dark matter candidate carried out with \ac{LUX} data.

Part~\ref{part:labwork}, Little Science, describes an R\&D test bed that was built at \ac{LBNL} over the course of this PhD and describes two studies completed with the test bed. 

Part~\ref{part:conclusion}, Conclusions, summarizes the work presented in this thesis and provides outlook for future experiments.


%*****************************************
%*****************************************
%*****************************************
%*****************************************
%*****************************************
