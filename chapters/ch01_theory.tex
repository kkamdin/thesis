%************************************************
\chapter{Theoretical Background }\label{theory} % $\mathbb{ZNR}$
%************************************************

%And is it over now, do you know how
%Pickup the pieces and go home.
%Fleetwood Mac, \textit{Rumours}, 1977
%Press your space face close to mine, love
%\begin{flushright}{\slshape    
%	I think, I'm looking on the dark side \\
%	But everyday you hurt my pride \\
%	I'm over my head \\
%	Oh, but it sure feels nice} \\ \medskip
%    --- {Fleetwood Mac, \textit{Over My Head}, 1975}
%\end{flushright}

\section{A Little History}
Our understanding of the universe develops in a leap-frog of theory and observation, one catching up to and surpassing the other as technology improves, to be passed in turn by a new idea or new observation. 


\section{The Standard Cosmology}
The standard cosmology is a parametrization of the Big-Bang cosmological model is also referred to as $\Lambda$CDM ("Lambda-CDM") and it accounts for the observed:

\begin{itemize}
\item cosmic microwave background (CMB)
\item large scale structure of galaxies and clusters
\item accelerating expansion of the universe
\item abundances of hydrogen and helium
\end{itemize}

Since the CMB was discovered in 1965 

\section{Evidence for Dark Matter}
\subsection{Galaxies and Clusters}
\subsection{Cosmic Microwave Background}

\section{Dark Matter Candidates Motivated by Particle Physics}
\subsection{WIMPS and the Hierarchy Problem}
\subsection{Axions and the Strong CP Problem}

\section{Motivation for LIPs}

\section{Experimental Strategies for Detecting Dark Matter}
\subsection{Production}

\subsection{Indirect Detection}

\subsection{Direct Detection}
%*****************************************
%*****************************************
%*****************************************
%*****************************************
%*****************************************
